\cleardoublepage

\chapter{Conclusiones y Trabajo Futuro}
\label{makereference6}

\subsection{Conclusiones}
\label{ref:Conclusions}

Con este trabajo he conseguido entender cómo funcionan la informática gráfica y
cómo utilizarla para visualización científica. He podido observar como es un
campo relativamente nuevo pero creciendo a una gran velocidad y ganando cada vez
más importancia.

He podido observar la enorme utilidad que puede llegar a tener la visualización
y la gran ayuda que puede suponer para los equipos que trabajan con grandes
volúmenes de datos. También he podido ser consciente como equipos científicos
enormemente importantes utilizan estas técnicas para realizar descubrimientos
pioneros, como es el caso del EHT~\cite{EHT} con la obtención de la primera imagen
de un agujero negro~\cite{1435174}.

\subsection{Posible Trabajo Futuro}
Como trabajo futuro se plantean varios puntos:

\begin{itemize}

		\item Seguir aprendiendo sobre el tema y alcanzar un mayor
				conocimiento que permita entender las últimas tendencias y
				descubrimientos en este área.

		\item Conocer más técnicas de visualización e incorporarlas a la
				aplicación.

		\item Mejorar la aplicación con la introducción de cuaterniones.

		\item Mejorar la aplicación con la posibilidad de realizar los tipos
				de visualización a partir de modelos y ficheros especificados
				por el usuario.

		\item Profundizar en la visualización de fluidos y en los nuevos
				métodos tridimensionales.

\end{itemize}
