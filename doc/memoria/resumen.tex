% +--------------------------------------------------------------------+ |
% Copyright Page
% +--------------------------------------------------------------------+

\newpage

\thispagestyle{empty}

\begin{center}

{\bf \Huge Resumen}

  \end{center} \vspace{1cm}

	En el mundo actual, las investigaciones y estudios científicos generan gran
	cantidad de datos que han de ser interpretados de una manera eficaz, con el
	fin de sacar las mejores conclusiones y obtener resultados fiables que no
	den lugar a la duda.  En este contexto, una de las disciplinas más
	importantes es la de la visualización, ya que puede ayudar a entender,
	ilustrar y obtener información relevante acerca del fenómeno que se está
	estudiando.

	De igual forma, en los últimos años se ha dado una expansión considerable de
	las capacidades de las GPUs, ofreciendo nuevas posibilidades dentro de
	la informática gráfica e incrementando el rendimiento tanto en computación
	paralela como en aplicaciones de visualización.
	
	En este trabajo se exploran estas ideas, haciendo hincapié en las
	posibilidades que nos ofrecen los distintos tipos de shaders gráficos dentro
	de la especificación OpenGL, y la manera en la que nos pueden ser útiles a
	la hora de interpretar datos y obtener representaciones para problemas
	típicos de visualización científica, como puede ser la visualización de
	datos en tres dimensiones, visualización de volúmenes, renderizado de curvas
	y superficies, etc.
	
\vspace{1cm}

% +--------------------------------------------------------------------+ | On
% the line below, replace Fecha |
% +--------------------------------------------------------------------+

\begin{center}

{\bf \Large Palabras clave}

   \end{center}

   \vspace{0.5cm}
   
   Visualización científica, GPU, Shader, OpenGL, Bézier, Superficies.
