% +--------------------------------------------------------------------+
% | Appendix A Page (Optional)                                         |
% +--------------------------------------------------------------------+

\cleardoublepage

\chapter{El lenguaje GLSL}
\label{ApendiceA}

En este apéndice se tratan algunas de las características del lenguaje GLSL,
sobre todo aquellas necesarias para la implementación de los shaders vistos en
la sección~\ref{makereference5.5}. Para una guía completa sobre el lenguaje se
puede consultar la documentación detallada~\cite{GLSLreference}.

\section{Versión}

Un programa GLSL ha de comenzar obligatoriamente con una declaración de la
versión de GLSL que se va a utilizar. Esto se realiza mediante una directiva que
se utiliza para saber la versión que se ha de utilizar para compilar el
programa. Esta directiva se declara del siguiente modo:

\begin{verbatim}
    #version 400
\end{verbatim}
En este caso se ha de utilizar la versión GLSL 4.0.

\section{Funciones y Estructuras de Control}

Como estructuras de control, GLSL soporta bucles y saltos, con las instrucciones
comunes del lenguaje C (if-else, switch, while, for\ldots). Sin embargo, la
recursión en este lenguaje no está permitida. 

En cuanto a las funciones, este lenguaje soporta tanto funciones definidas por
el usuario como funciones proporcionadas por el propio lenguaje. Una lista de
las funciones incluidas puede encontrarse en~\citet{GLSLreference}. De entre
estas funciones, se han utilizado las siguientes en los shaders desarrollados:

\begin{itemize}
		\item \verb|smoothstep| - Realiza una interpolación de Hermite entre dos
				valores~\cite{Hermite}.
		\item \verb|mix| - Realiza una interpolación lineal entre dos
				valores~\cite{LinearInterpolation}.
		\item \verb|normalize| - Calcula el vector unitario correspondiente a la
				dirección del vector original.
		\item \verb|max| - Devuelve el máximo entre dos valores.
		\item \verb|dot| - Calcula el producto escalar entre dos vectores.
		\item \verb|reflect| - Calcula la dirección de reflexión del vector
				incidente.
		\item \verb|pow| - Devuelve la potencia del primer parámetro elevado al
				segundo parámetro.
		\item \verb|EmitVertex| - Explicado en la sección~\ref{ref:GeoShader}.
		\item \verb|sin| - Devuelve el seno del parámetro.
		\item \verb|cos| - Devuelve el coseno del parámetro.
		\item \verb|texture| - Devuelve texels de una textura.
		\item \verb|clamp| - Restringir un valor para que se encuentre entre
				otros dos valores.
		\item \verb|discard| - Descarta el fragmento actual (Solo para el
				fragment shader).
\end{itemize}

\section{Tipos de Datos}

GLSL define una serie de tipos de datos. Algunos de estos son compartidos con
los lenguajes C y C++, mientras que otros son completamente nuevos. En nuestro
caso, los utilizados son los siguientes:

\subsection{Escalares}

\begin{itemize}
		\item \verb|bool| - Condicional. Valores \verb|true| o \verb|false|.
		\item \verb|int| - Entero con signo de 32 bits.
		\item \verb|float| - Número de coma flotante.
\end{itemize}

\subsection{Vectoriales}

GLSL soporta tipos de datos vectoriales de $n$ componentes, con $n$ siendo $2,
3$ o $4$. Los tipos vectoriales soportan las mismas operaciones que los
escalares. Estas operaciones se realizan componente a componente. Sin embargo,
estas operaciones funcionarán solo si los dos vectores tienen el mismo número de
componentes.

\begin{itemize}
		\item \verb|vec|$n$ - Un vector de $n$ componentes tipo
				\verb|float|.
\end{itemize}

\subsection{Swizzling}

El swizzling consiste en la posibilidad que ofrece GLSL de acceder a las
componentes de un vector de la siguiente manera:

\begin{verbatim}
    vec4 someVec;
    someVec.x + someVec.y;
\end{verbatim}

Se pueden utilizar \verb|x|, \verb|y|, \verb|z| y \verb|w| para acceder a las
componentes primera, segunda, tercera y cuarta respectivamente. Además se puede
construir un nuevo vector a partir de otro especificando el orden de las
coordenadas del vector antiguo en el nuevo. Con todo esto, algunos ejemplos más
ilustrativos del swizzling son los siguientes:

\begin{verbatim}
    vec2 someVec;
    vec4 otherVec = someVec.xyxx;
    vec3 thirdVec = otherVec.zyy;	
    vec4 someVec;
    someVec.wzyx = vec4(1.0, 2.0, 3.0, 4.0); // Da la vuelta al vector.
    someVec.zx = vec2(3.0, 5.0); // Pone la componente z a 3 y la x a 5.
\end{verbatim}

Además, para facilitar el acceso a un vector que represente un color o a uno que
represente coordenadas  de una textura, se pueden utilizar, juntos a las letras
anteriores, las siguientes combinaciones:

\begin{itemize}
		\item \verb|rgba|
		\item \verb|stpq|
\end{itemize}

Estas combinaciones no suponen ninguna diferencia con la principal, son
simplemente azúcar sintáctico para ayudar al programador. Sin embargo, no se
pueden realizar combinaciones que contengan letras de distintos tipos. Por
ejemplo, la combinación: 
\begin{verbatim}
    vec3 aVec = someVec.xbg;
\end{verbatim}
resultaría en un error de compilación.

\subsection{Matrices}

También soporta tipos matriciales. Para declarar una matriz se utiliza la
siguiente sintaxis:

\begin{itemize}
		\item \verb|mat|$n$\verb|x|$m$ - Matriz $n \times m$
		\item \verb|mat|$n$ - Matriz $n \times n$
\end{itemize}	
donde $n$ puede ser $2, 3$ o $4$. En los tipos matriciales no se puede utilizar
swizzling, por lo que han de ser accedidos mediante una sintaxis de array propia
de C o C++.

\subsection{Uniforms}

Los \verb|uniform| son un tipo de variable de shader. Se utilizan como
parámetros que el programador de un shader puede pasar al programa principal. Se
llaman de esta manera porque su valor no cambia de una invocación de shader a
otra durante la misma llamada de renderizado. 

Estas variables han de ser declaradas de un modo global dentro del shader y
pueden ser de cualquiera  de los tipos permitidos en GALS. Son variables  de
solo lectura dentro del shader, y cualquier intento de cambio de su valor
resultará en un error de compilación. Están pensados para utilizarse en el
programa desde OpenGL y no desde el shader.

Para utilizar una variable \verb|uniform| desde la aplicación, se ha de recabar
la localización del shader mediante una llamada a la función
\verb|glGetUniformLocation|.

\section{Variables y Entrada/Salida}

Los shaders han de ser capaces de comunicarse entre si dentro del pipeline. Las
entradas y salidas de cada uno de los tipos de shader se han explicado en la
sección~\ref{makereference3}. En esta sección se expone la sintaxis necesarias
para declarar estas variables de entrada y salida desde cada tipo de shader.

Asimismo, existe una serie de variables específicas construidas dentro del
lenguaje GLSL que tienen un significado ya definido.

\subsection{Entradas y salidas}

De manera general, una variable de entrada a un shader ha  de llevar el
modificador \verb|in|. De igual modo, una variable de salida dentro de un shader
ha de llevar el modificador \verb|out|. Por tanto, si queremos pasar un valor
desde el vertex shader al fragment shader (suponiendo que no se utilizan
tessellation ni geometry shaders) declararíamos lo siguiente en el vertex shader:

\begin{verbatim}
    out vec3 vPos;	
\end{verbatim}

Y, de manera correspondiente, lo siguiente en el fragment shader:

\begin{verbatim}
    in vec3 vPos;	
\end{verbatim}

Un caso particular proviene de las entradas del vertex shader. Al ser el primero
en el pipeline, las entradas a este shader provienen de los datos contenidos en
un VBO. Estos datos corresponden a los diferentes atributos de un vértice en
particular, y están referenciados en un VAO. Así, el primer atributo de un
vértice, especificado en el VAO, puede ser la posición, mientras que el segundo
puede ser el color. Para pasar esta información al shader se declaran las
siguientes líneas dentro del shader:

\begin{verbatim}
    layout(location = 1) in vec3 aPos;		
    layout(location = 2) in vec3 aColor;		
\end{verbatim}

Esto quiere decir al shader que en el primer atributo ha de encontrar
un tipo \verb|vec3| que representa la posición y en el segundo atributo ha de
encontrar otro \verb|vec3| que representa el color del vértice correspondiente.

De igual modo se pueden especificar salidas desde la última etapa del pipeline,
el fragment shader.

Además de estos dos casos particulares, se dan los de aquellos shaders que
necesitan instrucciones específicas para realizar su función. Son los
siguientes:

\subsubsection{TCS}

Para utilizar el shader de teselación, se ha de habilitar una extensión ARB, que
se realiza mediante la siguiente directiva.

\begin{verbatim}
    #extension GL_ARB_tessellation_shader : enable
\end{verbatim}

Además el TCS, acorde a lo expuesto en la seccion~\ref{ref:TesConShader}, ha de
especificar el número de vértices en el parche de salida. Esto no se realiza
mediante variables específicas, sino mediante la siguiente instrucción:

\begin{verbatim}
    layout( vertices = 16 ) out;
\end{verbatim}

\subsubsection{TES}

En este caso, al igual que en el TCS, se ha de habilitar la extensión ARB que
permite la utilización de tessellation shaders, por lo que se ha de incluir la
misma línea. 

Además, en este caso se ha de especificar lo explicado en la
sección~\ref{ref:TesEvaShader} (espaciado, primitiva, orden), que se realiza
mediante la siguiente instrucción:

\begin{verbatim}
    layout( quads, equal_spacing, ccw ) in;
\end{verbatim}

\subsubsection{Geometry shader}

Para habilitar la utilización de este shader se ha de habilitar otra extensión
ARB:

\begin{verbatim}
    #extension GL_EXT_geometry_shader4: enable
\end{verbatim}

Además se ha de definir tanto la primitiva de entrada como la de salida y el
número máximo de vértices que se pueden generar (Sección~\ref{ref:GeoShader}).
Para ello hay que utilizar las siguientes líneas:

\begin{verbatim}
    layout( lines_adjacency ) in;
    layout( line_strip, max_vertices=256 ) out;
\end{verbatim}

\subsection{Variables Específicas}

Además de las variables definidas por el usuario y de las instrucciones
específicas necesarias anteriores, GLSL tiene algunas variables ya definidas que
se utilizan para propósitos específicos, aunque no es necesario utilizarlas.
Estas variables pueden ser específicas para cada tipo de shader o comunes a
todos ellos. Una lista de estas variables puede encontrarse
en~\citet{GLSLreference}. Aquí se muestran las utilizadas en la aplicación.

\subsubsection{Vertex Shader}

\begin{itemize}
		\item \verb|out vec4 gl_Position| - La posición del vértice actual en el
				clip space.
		\item \verb|out float gl_PointSize| - El tamaño del punto siendo
				rasterizado. Solo funciona con la primitiva \verb|GL_POINTS|.
\end{itemize}

\subsubsection{TCS}

\begin{itemize}
		\item \verb|in int gl_InvocationID| - El índice de la invocación del TCS
				dentro del patch. El TCS escribe a las variables de salida
				vértice a vértice utilizando esta variable para indexarlos.
		\item \verb|patch out float gl_TessLevelOuter[4]| - Niveles externos de
				teselación.
		\item \verb|patch out float gl_TessLevelInner[2]| - Niveles internos de
				teselación.
		\item \verb|gl_in[gl_MaxPatchVertices]| - Información de cada vértice
				proveniente del vertex shader.
\end{itemize}

\subsubsection{TES}

\begin{itemize}
		\item \verb|in vec3 gl_TessCoord| - La localización dentro del abstract
				patch para cada vértice particular.
		\item \verb|gl_in[gl_MaxPatchVertices]| - Información de cada vértice
				proveniente del TCS.
\end{itemize}

\subsubsection{Geometry Shader}

\begin{itemize}
		\item \verb|gl_Position| - La posición del vértice actual dentro del
				clip space.
\end{itemize}
