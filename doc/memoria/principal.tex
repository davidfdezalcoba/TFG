% +--------------------------------------------------------------------+
% | The LaTex keyword \documentclass selects a particular class to     |
% | associate with the document.  The current documentclass            |
% | {class_diss} generates a Table of Contents that has leading dots   |
% | only on chapter subheadings.  If you prefer a Table of Contents    |
% | that has leading dots for all entries, replace {class_diss}        |
% | with {Mydiss} in the command below.                                |
% |                                                                    |
% +--------------------------------------------------------------------+

\documentclass[final,12pt,oneside]{class_diss}
\bibliographystyle{unsrtnat}

\usepackage[utf8]{inputenc}
\usepackage[T1]{fontenc}
\usepackage[spanish,es-tabla]{babel}

\usepackage{caption}
\usepackage{subcaption}
\usepackage{array}
%\usepackage{     caption2} % Customize captions a bit more
\usepackage{      amsmath} % American Mathematics Society standards
%\usepackage{      wrapfig} % Wraps text around a figure or table
\usepackage{     graphicx} % Extended graphics package.
%\usepackage{     fancyhdr} % Efficiently handles headers and footers
%\usepackage{       braket} % Bra-Ket notation package
%\usepackage{     mathrsfs} % Specialized Math fonts (Hamiltonian, etc.)
%\usepackage{boxedminipage} % Boxed text can be produced
%\usepackage{     setspace} % Controls line spacing via \begin{space}

\usepackage{amsxtra}
\usepackage{amssymb}
\usepackage{amsthm}
\usepackage{latexsym}

\usepackage[usenames]{color}

\definecolor{  Pink}{rgb}{1.0, 0.5, 0.5}
\definecolor{Maroon}{rgb}{0.8, 0.0, 0.0}

\usepackage[sort&compress]{natbib}
\bibpunct{[}{]}{,}{n}{}{}
\usepackage{hypernat}

% Hyperref settings

\usepackage[pdftex, plainpages=false, pdfpagelabels]{hyperref}

\hypersetup{
    linktocpage=true,
    colorlinks=true,
    bookmarks=true,
    citecolor=blue,
    urlcolor=red,
    linkcolor=Maroon,
    citebordercolor={1 0 0},
    urlbordercolor={1 0 0},
    linkbordercolor={.7 .8 .8},
    breaklinks=true,
    pdfpagelabels=true,
}

% Margins

\topmargin      = -0.56in
\textheight     =  8.60in
\textwidth      =  6.46in
\oddsidemargin  =  0.02in

% Beginning of document

\begin{document}

  \setcounter{page}{-1}

% +--------------------------------------------------------------------+
% | Title Page
% +--------------------------------------------------------------------+

\newpage

% No page number

\thispagestyle{empty}

\begin{center}

   \vspace{1cm}

   {\Large GRAPHICS SHADERS FOR SCIENTIFIC VISUALIZATION}\\

   \vspace{0.5cm}

   \vspace{0.5cm}

   {\large DAVID FERNÁNDEZ ALCOBA}\\

   \vspace{0.5cm}

   DOBLE GRADO EN INGENIERÍA INFORMÁTICA - MATEMÁTICAS\\ 
   FACULTAD DE INFORMÁTICA\\
   UNIVERSIDAD COMPLUTESNE DE MADRID \\

   \vspace{0.65cm}
   \rule{2in}{0.5pt}\\
   \vspace{1.85cm}

  \includegraphics[height=2.5in]{figures/escudo.jpg}
  
   \vspace{1.5cm}
Trabajo de Fin de Grado en Ingeniería Informática - Matemáticas

   \vspace{0.5cm}

  \today\\
   \vspace{1cm}

\end{center}

{\raggedleft
   \vspace{ 1cm}
Directora:\\
   \vspace{ 0.2cm}
Ana Gil Luezas\\
}

   \pdfbookmark[0]{Portada}{PDFPortadaPage}

% +--------------------------------------------------------------------+
% | Copyright Page
% +--------------------------------------------------------------------+

\newpage

\thispagestyle{empty}

\begin{center}

{\bf \Huge Autorización de difusión}

\vspace{1cm}

% +--------------------------------------------------------------------+
% | On the line below, replace "Enter Your Name" with your name
% | Use the same form of your name as it appears on your title page.
% | Use mixed case, for example, Lori Goetsch.
% +--------------------------------------------------------------------+

   \large David Fernández Alcoba

   \vspace{0.5cm}

% +--------------------------------------------------------------------+
% | On the line below, replace Fecha
% |
% +--------------------------------------------------------------------+

   \today

   \vspace{0.5cm}
   \end{center}
   
El/la abajo firmante, matriculado/a en el Doble Grado en Ingeniería Informática
y Matemáticas de la Facultad de Informática, autoriza a la Universidad
Complutense de Madrid (UCM) a difundir y utilizar con fines académicos, no
comerciales y mencionando expresamente a su autor el presente Trabajo Fin de
Grado “GRAPHIC SHADERS FOR SCIENTIFIC VISUALIZATION”, realizado durante el curso
académico 2018-2019 bajo la dirección de Ana Gil Luezas en el Departamento de
Sistemas Informáticos y Computación, y a la Biblioteca de la UCM a depositarlo
en el Archivo Institucional E-Prints Complutense con el objeto de incrementar la
difusión, uso e impacto del trabajo en Internet y garantizar su preservación y
acceso a largo plazo.

	\vspace{2.5cm}
	\rightline{David Fernández Alcoba}

   \pdfbookmark[0]{Autorización}{PDFAutorizacionPage}

% +--------------------------------------------------------------------+ |
% Copyright Page
% +--------------------------------------------------------------------+

\newpage

\thispagestyle{empty}

\begin{center}

{\bf \Huge Resumen}

  \end{center} \vspace{1cm}

    Tradicionalmente, la visualización científica se ha realizado en CPUs. Sin
    embargo, vivimos unos días en los que las GPUs están mejorando
    drásticamente, por lo que resulta natural una transición de la visualización
    científica hacia entornos que utilicen las capacidades de paralelización que
    nos proporcionan las GPUs. En este contexto, los shaders gráficos,
    introducidos en las tarjetas gráficas y programables, nos aportan
    capacidades muy interesantes para este ámbito. Así, en este texto se
    analizan y describen las capacidades de cada uno de los tipos de shaders del
    pipeline de renderizado aplicadas a técnicas de visualización tradicionales,
    como transformaciones de imágenes, nubes de puntos para datos escalares 3D,
    visualización de alturas de terrenos, curvas y superficies de bézier o
    hedgehog plots para datos vectoriales en 3D.

\vspace{1cm}

% +--------------------------------------------------------------------+ | On
% the line below, replace Fecha |
% +--------------------------------------------------------------------+

\begin{center}

{\bf \Large Palabras clave}

   \end{center}

   \vspace{0.5cm}
   
   Visualización científica, GPU, Shader, Bézier
   



   \pdfbookmark[0]{Resumen}{PDFResumenPage}

% +--------------------------------------------------------------------+
% | Copyright Page
% +--------------------------------------------------------------------+

\newpage

\thispagestyle{empty}

\begin{center}

{\bf \Huge Abstract}

  \end{center}
\vspace{1cm}

	Scientific visualization has traditionally been running on CPUs. However, we
	live in a time where GPUs are improving at a drastic rate, making a
	transition toward the parallelization capabilities that GPUs provide seem
	natural.  Within this context, graphic shaders, introduced in graphics cards
	and programmable, give very interesting features for us to use in scientific
	visualization. This text analyzes and describes the various properties of
	each of the shaders in the graphics pipeline, applying them to several
	well-known scientific visualization techniques. Some of the techniques
	covered involve image transformation, point cloud for visualizing 3D scalar
	data, height coloring terrains, bézier curves and surfaces or hedgehog plots
	for 3D vectorial data.

\vspace{1cm}

% +--------------------------------------------------------------------+
% | On the line below, replace Fecha
% |
% +--------------------------------------------------------------------+

\begin{center}

{\bf \Large Keywords}

   \end{center}

   \vspace{0.5cm}
   
	Scientific Visualization, Graphic Shaders, GPU, Bézier
   



	\pdfbookmark[0]{Abstract}{PDFAbstractPage}

 \vfill

% +--------------------------------------------------------------------+
% | We use the following code to suppress page numbers and other
% | style issues we do not want present on a given page.               |
% +--------------------------------------------------------------------+

%\thispagestyle{empty} Looks like it's ok to remove this line
\newpage
\pagenumbering{roman}

% +--------------------------------------------------------------------+
% | On the line below, set the number to represent the page number of
% | the Table of Contents page.  For example, if the Table of Contents
% | page is the 8th page of your document, enter 8 in the brackets.  This
% | number may vary, depending on the length of your abstract.
% |
% | Numbers do not appear on the title and abstract pages, but they are
% | included in the page count.  The Table of Contents page is the
% | first page on which page numbers are displayed.
% +--------------------------------------------------------------------+

\setcounter{page}{5}

% +--------------------------------------------------------------------+
% | Here, we will generate our Table of Contents (TOC) entries.        |
% | This adds the section to the TOC and then generates the indicated  |
% | section.                                                           |
% +--------------------------------------------------------------------+

\phantomsection
\addcontentsline{toc}{chapter}{Índice}

\tableofcontents
\listoffigures
\listoftables

\hfill  %Are these lines necessary?
\hfill

% +--------------------------------------------------------------------+
% | We use arabic (1, 2, 3...) page numbering starting from page 1.    |
% | Note, however, that there are many pages where this is not the     |
% | desired behavior - such as the Title page, or abstract.  In these  |
% | cases, we can use \thispagestyle{empty} to suppress page numbers,  |
% | and other general style issues that we've defined globally.        |
% +--------------------------------------------------------------------+

\newpage
\pagenumbering{arabic}
\setcounter{page}{1}

% +--------------------------------------------------------------------+
% | Here is where we include individual sections of the thesis or
% | dissertation.                                                      |
% +--------------------------------------------------------------------+

% +--------------------------------------------------------------------+
% | Chapters
% +--------------------------------------------------------------------+

% +--------------------------------------------------------------------+
% | Sample Chapter
% |
% | This file provides examples of how to
% | - insert a figure with a caption
% | - construct a table with a caption
% | - create subsections within the chapter
% | - insert a reference to a Figure or Table
% | - make a citation
% +--------------------------------------------------------------------+

\cleardoublepage

% +--------------------------------------------------------------------+
% | Replace "Chapter Title" below with the title of your chapter.  LaTeX
% | will automatically number the chapters.
% +--------------------------------------------------------------------+

\chapter{Introducción}
% \label{ch:chapter1}
\label{makereference}

Tal y como se cuenta en~\citet{DEFANTI1991247}, los científicos computacionales
basan su trabajo en fuentes de datos de gran volumen. Sin embargo, estos datos
tienen tal magnitud que los científicos se ven, a menudo, superados. Entre las
fuentes de datos de gran volumen se encuentran:

\begin{itemize}
		\item Supercomputadores
		\item Inteligencia militar, satélites, datos astronómicos y de tiempo atmosférico
		\item Sondas enviando datos desde otros planetas
		\item Radio telescopios terrestres
		\item Instrumentos capturando temperaturas oceánicas, movimientos tectónicos y 
				actividad volcánica y sísmica
		\item Escáneres médicos empleando distintas técnicas de imagen como tomografía, 
				resonancias magnéticas, etc
\end{itemize}

Simplemente con un formato numérico, el cerebro humano es incapaz de interpretar
gigabytes de datos cada día, resultando en mucha información desperdiciada. De
aquí surge la necesidad de una alternativa a los números. La posibilidad de los
científicos para visualizar cómputos complejos y simulaciones es absolutamente
esencial para asegurar la integridad de análisis y predicciones, así como
presentar esta información al resto.\\

Esta capacidad de visualización se hace especialmente importante en el ser
humano, puesto que, de todas nuestras funciones cerebrales, nuestro sistema de
visión es el que mayor capacidad de procesamiento de información tiene. Según
expertos en conocimiento, el procesamiento de información en humanos tiene dos
formas: preconsciente y consciente. El procesamiento de información
preconsciente es involuntario, similar a la respiración. Este es el tipo de
procesamiento que se da en información gráfica.
~\citet{Rohrer:2000:SBI:510378.510552}

Teniendo esto en cuenta y el hecho de que cada persona tiene una capacidad de
vision espacial diferente, la informática gráfica puede ayudar a aquellos que
tienen una mayor dificultad y que, de otro modo, serían incapaces de visualizar
conceptos complejos.

Estos hechos muestran una necesidad ha resultado en el surgimiento, en la última
década, de una disciplina totalmente independiente, la visualización científica. 

\section{Motivación}
\label{makereference1.1}

La importancia de lo expuesto anteriormente sirve como suficiente motivación,
aunque a esto se ha de añadir el reto personal de, con este trabajo, aprender
y entender un área de la informática que no forma parte del itinerario en mi
formación, como es la informática gráfica, y que engloba muchas de las materias
vistas hasta ahora tanto en ingeniería informática como en matemáticas.\\

Además, esta rama dentro de la investigación científica es relativamente
reciente, asociándose su nacimiento en 1987 al
artículo~\citet{McCormick:1988:VSC:43965.43966}, por lo que aún hay muchos retos
y problemas por resolver, haciendo su estudio muy interesante.

\section{Objetivos}
\label{makereference1.2}

El objetivo principal de este trabajo es el de aprender el funcionamiento básico
de los gráficos y la aplicación de éstos a la investigación científica.
Este objetivo se puede desglosar en otros subobjetivos más concretos y que
marcan la línea de trabajo:
\begin{enumerate}
		\item Comprender el pipeline de gráficos y la utilidad y funcionamiento
				de los shaders, así como aprender el lenguaje GLSL para su
				escritura.
		\item Aprender las técnicas más conocidas de visualización científica y
				cómo desarrollar shaders que las implementen.
		\item Desarrollar una aplicación que ponga de manifiesto lo aprendido,
				desarrollando shaders que ilustren algunas de las técnicas
				vistas.
\end{enumerate}

\section{Plan de trabajo}
\label{makereference1.3}
Con estos objetivos en mente, se desarrolló el siguiente plan de trabajo,
acordado en reuniones iniciales entre tutora y autor del trabajo.

\begin{itemize}
		\item \textbf{Toma de contacto con OpenGL} Durante esta fase se leyeron
				tutoriales sobre OpenGL y se experimentó con diversos shaders y
				librerías para familiarizarse con la tecnología, a la vez que se
				aprendía el lenguaje GLSL.
		\item \textbf{Documentación} Durante la duración completa del proyecto
				se llevó a cabo una documentación acerca de las distintas
				fuentes de información, con el objetivo de no olvidar incluir
				partes importantes en la memoria.
		\item \textbf{Comunicación con el tutor} Se concretaron diversas
				reuniones con la tutora durante las partes intensivas del
				proyecto con el fin de mostrar avances y acordar los siguientes
				pasos. Asimismo, se mantuvo una comunicación mediante correo
				electrónico para aquellas dudas menores que surgieron durante la
				realización del trabajo.
		\item \textbf{Preparación del entorno de desarrollo} Durante esta fase
				se preparó el equipo, instalando las librerías y programas
				necesarios para el correcto funcionamiento de la aplicación.
		\item \textbf{Desarrollo de la aplicación} Una vez preparado el entorno,
				se continuó durante toda la duración del trabajo con el
				desarrollo de la aplicación, incluyendo cada vez nuevas
				capacidades.
		\item \textbf{Redacción de la memoria} Se inició la redacción de la
				memoria una vez se tenían conocimientos suficientes, a mitad de
				la elaboración del trabajo. Una vez comenzada la redacción, se
				fue reeditando y mejorando en un proceso iterativo.
\end{itemize}

\section{Estructura de la memoria}
\label{makereference1.4}

El siguiente capítulo, \textbf{OpenGL y DirectX}~\ref{makereference2}, presenta
las dos grandes especificaciones dentro de la informática gráfica, centrándose
en OpenGL y analizando sus características, capacidades y debilidades, así como
las diferencias entre ambas.\\

Posteriormente, el capítulo \textbf{Shaders y Visualización
Científica}~\ref{makereference3} explora las técnicas más comunes dentro del
campo de visualización científica y qué tipos de shaders son útiles para cada
una de ellas, introduciendo algunos de los que más adelante se presentarán junto
a la aplicación.\\

En el capítulo \textbf{Aplicación}~\ref{makereference4} se presenta la
aplicación desarrollada, explicando el diseño, capacidades, experimentos
realizados\ldots\\

Por último el capítulo \textbf{Conclusiones y Trabajo
Futuro}~\ref{makereference5} incluye un análisis del trabajo realizado, el nivel
de cumplimiento de los objetivos propuestos y posibles líneas de trabajo
futuro.\\

\begin{table}[b]
	% \begin{center}
		\begin{tabular}[b]{|c|}
			\hline
			\\
			El código de la aplicación desarrollada puede encontrarse en el
			siguiente enlace:\\
			http://github.com/davidfdezalcoba/TFG\\
			\\
			\hline
		\end{tabular}
	% \end{center}
\end{table}

% How to insert figures:
% \begin{figure}[htb]%t=top, b=bottom, h=here
% 
% 	\begin{center}
% 	    \includegraphics[height=2.5in]{figures/lowressur.png}
% 	    \includegraphics[height=2.5in]{figures/highressur.png}
% 	\end{center}
% 
%     \caption[Optional: Short caption to appear in List of
%     Figures]{Full caption to appear below the Figure}
% 
%     \label{figure1}
% 
% \end{figure}

% +--------------------------------------------------------------------+
% |To create cross-references to figures, tables and segments
% |of text, LaTeX provides the following commands:
% |   \label{marker}
% |   \ref{marker}
% |   \pageref{marker}
% | where {marker} is a unique identifier.
% |
% | In the line above, we use \label{figure1} to mark a location
% | we wish to refer to later.  LATEX replaces \ref by the number of
% | the chapter, section, subsection, figure, or table after which the
% | corresponding \label command was issued. \pageref prints the page
% | number of the page where the \label command occurred.
% |
% +--------------------------------------------------------------------+

% +--------------------------------------------------------------------+
% | The table is created with this command
% |
% | \begin{tabular}[pos]{table spec}
% |
% | The "pos" argument specifies the vertical position of the table relative to
% | the baseline of the surrounding text.  Use t, b, or c to specify alignment
% | at the top, bottom, or center.
% |
% | The "table spec" command defines the format of the table
% |   l for a column of left-aligned text
% |   r for a column of right-aligned text
% |   c for centered text
% |   p{width} for a column containing justified text with line breaks
% |   | for a vertical line
% +--------------------------------------------------------------------+

% +--------------------------------------------------------------------+
% | Replace \section headings below with the title of your
% | subsections.  LaTeX will automatically number the subsections 1.1,
% | 1.2, 1.3, etc.
% +--------------------------------------------------------------------+

% +--------------------------------------------------------------------+
% | Some insight about citations

% | In this paragraph, we want to refer to Fig.~\ref{figure1}
% | mentioned at the beginning of this chapter.  We also refer to the
% | Table~\ref{table1}.
% | 
% | \section{Making a Reference to a Chapter Subsection}
% | \label{makereference1.2}
% | 
% | In this section, we refer back to text mentioned in
% | Section~\ref{makereference1.1} on page~\pageref{makereference1.1}.
% +--------------------------------------------------------------------+

% +--------------------------------------------------------------------+
% | Sample Chapter 2
% +--------------------------------------------------------------------+

\cleardoublepage

% +--------------------------------------------------------------------+
% | Replace "This is Chapter 2" below with the title of your chapter.
% | LaTeX will automatically number the chapters.
% +--------------------------------------------------------------------+

\chapter{OpenGL y DirectX}
\label{makereference2}

El objetivo de este capítulo es explicar en qué consiste OpenGl, así como su
pipeline de gráficos, los tipos de shaders que incluye y las diferencias que
presenta con DirectX, su principal competidor.\\

\section{¿Qué es?}
\label{makereference2.1}

OpenGL se define como una API \textit{(application programming interface)}, que
es simplemente una librería de software para acceder a capacidades del hardware
de gráficos (ver \citet{Shreiner:2009:OPG:1696492}).

OpenGL está diseñado como una interfaz independiente del hardware que puede ser
implementada en muchos sistemas hardware de gráficos diferentes, o completamente
como software, en el caso de que el sistema no posea hardware de gráficos.
OpenGL no proporciona ninguna funcionalidad para describir modelos en tres
dimensiones ni operaciones para leer ficheros (como imágenes JPEG, por ejemplo).
En su lugar, se deben construir los objetos tridimensionales a partir de un
pequeño conjunto de primitivas geométricas---puntos, líneas, triángulos y
parches. 

\section{Breve historia de OpenGL}
\label{makereference2.2}

OpenGL nace a principios de los años 90, desarrollada por Silicon Graphics~(SGI).  En
los años 80, Silicon Graphics poseía una API privada denominada IRIS GL,
utilizada para producir gráficos en sus estaciones de trabajo IRIS.
Posteriormente, debido a la pérdida de cuota de mercado, decidió hacer su API
pública. Sin embargo, a causa de problemas con patentes y el hecho de tener
características poco relevantes para los gráficos 3D como la funcionalidad de
ventanas, se decidió reescribir algunas de las partes y se lanzó lo que ahora se
conoce como OpenGL.\\

Esta nueva especificación consiguió logros importantes para la informática gráfica,
como estandarizar el acceso al hardware gráfico, trasladar a los fabricantes la
responsabilidad del desarrollo de las interfaces con el hardware y delegar la
funcionalidad de ventanas al sistema operativo. Todo esto supuso un gran impacto
en la industria, al ofrecer a los desarrolladores una plataforma de alto nivel
sobre la que trabajar.\\

En 1992, Silicon Graphics lideró la creación del OpenGL Architecture Review
Board (OpenGL ARB)~\cite{OpenGLARB}, un grupo de empresas del sector que sería la encargada de
mantener y extender la especificación en los años siguientes. El OpenGL ARB
estaba formado por 3Dlabs, Apple, ATI, Dell, IBM, Intel, Nvidia, SGI and Sun
Microsystems.\\

En otoño de 2006, el OpenGL ARB y los directores de Khronos votaron para transferir el
control de OpenGL a Khronos Group. El secretario de la ARB Jon Leech observó:
\textit{"Hemos decidido mover OpenGL a Khronos para asegurar la salud futura de
OpenGL en todas sus formas."}\cite{OpenGLARB}

\section{Diseño}
\label{makereference2.3}

\begin{figure}[h]%t=top, b=bottom, h=here

	\begin{center}
	    \includegraphics[height=4in]{figures/pipeline.png}
	    \includegraphics[height=4in]{figures/pipelineExtendido.png}
	\end{center}

    \caption[Pipeline de OpenGl]{Pipeline de OpenGl en el hardware gráfico.}
	\label{figurePipeline}

\end{figure}



% +--------------------------------------------------------------------+
% | In this chapter, I want to refer to Chapter~\ref{makereference},
% | so I'm going to use the slash ref command along with the
% | "makereference" label which I assigned back at the beginning of
% | Chapter 1.
% | 
% | \section{Page Number References}
% | \label{makereference2.1} I should also be able to refer to a
% | specific page number, such as page~\pageref{makereference}.  Of
% | course, I'll need to have a slash label command and a unique name
% | in each section that I want to be able to refer to later in the
% | text.
% | 
% | \section{Referring to Sections Within Chapter 1}
% | \label{makereference2.2} Now, I'm going to refer to different
% | sections within Chapter 1. I gave an example of a figure in
% | section~\ref{makereference1.1} and an example of a table in
% | section~\ref{makereference1.2}.  In
% | section~\ref{makereference1.3}, we looked at examples of
% | bibliographic citations.
% +--------------------------------------------------------------------+

% +--------------------------------------------------------------------+
% | Sample Chapter 3
% +--------------------------------------------------------------------+

\cleardoublepage

% +--------------------------------------------------------------------+
% | Replace "This is Chapter 3" below with the title of your chapter.
% | LaTeX will automatically number the chapters.
% +--------------------------------------------------------------------+

\chapter{Shaders y Visualización Científica}
\label{makereference3}

Como se ha visto en la sección~\ref{makereference2.3}, el pipeline de gráficos de
OpenGL tiene cuatro etapas programables:

\begin{itemize}
		\item Shader de Vértices (Vertex Shader)
		\item Shaders de Teselación
				\subitem Shader de Control de Teselación (Tessellation Control
				Shader)
				\subitem Shader de Evaluación de Teselación (Tessellation
				Evaluation Shader)
		\item Shader Geométrico (Geometry Shader)
		\item Shader de Fragmento (Fragment Shader)
\end{itemize}

En este capítulo se explicará cómo funcionan, cómo desarrollarlos y cómo
utilizarlos para resolver problemas de visualización científica habituales.

\section{Shaders}
\label{makereference3.1}

Los shaders gráficos son un tipo de programa utilizado inicialmente para
producir niveles apropiados de luz, oscuridad y color en una imagen. Sin
embargo, hoy en día se utilizan con diversas finalidades diferentes como efectos
especiales, post procesado de vídeos, videojuegos, etc.\\

Los shaders se introdujeron en OpenGL en la versión 2.0, incluyendo el lenguaje
de programación centrado en shaders OpenGL Shading Language, también conocido
como GLSL~\cite{GLSL}--- un lenguaje tipo C creado específicamente para que los
desarrolladores tuviesen más control sobre el pipeline de renderizado.---\\

Durante el proceso de desarrollo de shaders no es necesario incluir todas las
etapas que se muestran en la Figura~\ref{fig2.2b}, aunque normalmente, si se
decide utilizar alguno de ellos, se requiere utilizar, al menos, un Vertex
Shader. 

Los shaders del pipeline se comunican entre ellos mediante variables
proporcionadas por GLSL, siendo la salida de un shader la entrada del siguiente,
como se muestra en la Figura~\ref{fig3.1}. Un breve resumen acerca del lenguaje
GLSL se incluye en el Apéndice~\ref{ApendiceA}.

\begin{figure}
		\centering
		\includegraphics[height=10cm]{figures/variablespipe.png}
		\caption{Comunicación entre shaders del pipeline}
		\label{fig3.1}
\end{figure}

\subsection{Vertex Shader}
\label{ref:Vertex}

El Vertex Shader es la etapa de sombreado en el pipeline de renderizado que se
encarga del procesamiento de los vértices individuales~\cite{VertexShader}. El
Vertex Shader tiene como entrada unos atributos de vértice especificados desde
un \textit{Vertex Array Object (VAO)} por un comando de dibujo. Recibe un único
vértice, formado por sus atributos, del flujo de vértices y genera un único
vértice al flujo de salida. Por cada vértice de entrada ha de haber,
necesariamente, uno de salida. \\

Este shader se invoca una vez por cada vértice en el flujo de entrada,
exceptuando el caso en el que OpenGL detecte que una invocación a este shader
con las exactamente las mismas entradas ya ha sido realizada, en cuyo caso se
reutilizan los resultados de la invocación previa, resultando en un ahorro
de tiempo valioso. \\

Normalmente, las operaciones que se realizan en el Vertex Shader son
transformaciones para el espacio de post-proyección, iluminación por vértice o
preparación para las siguientes etapas del pipeline. \\

\subsection{Tessellation Shaders}
\label{ref:TesShaders}

La Teselación es la etapa, opcional, del pipeline de renderizado que consiste en
subdividir un parche de algún tipo y computar los valores de los nuevos vértices
creados en el proceso. Está compuesta, a su vez, por otras tres etapas, dos de
ellas programables en forma de shader, y una intermedia fija. Cada una de estas
etapas se encarga de una parte del proceso de teselación.\\ 

En esta sección se explican los dos shaders involucrados, además de la etapa
intermedia, llamada generador de primitivas de teselación, pues resulta
importante para entender el proceso y las entradas y salidas a los shaders.

\subsubsection{Tessellation Control Shader}
\label{ref:TesConShader}

El Tessellation Control Shader (TCS)~\cite{TesConShader} es la primera etapa del
proceso de teselación, en el caso de ser utilizado. Se sitúa inmediatamente
posterior al Vertex Shader e inmediatamente anterior al generador de primitivas
de teselación. Controla cuánta teselación provocar en un parche determinado, así
como el tamaño del parche, permitiendo aumentar la cantidad de datos. Su función
principal es la de comunicar al generador de primitivas de teselación el nivel
de teselación deseado, así como proveerle los datos del parche al Tessellation
Evaluation Shader mediante sus variables de salida. \\

Como entrada, el TCS obtiene la salida del Vertex Shader organizada en un vector
de tantos vértices como tenga el parche de entrada. Cada invocación al TCS
produce un único vértice como salida al parche de salida. Por cada vértice en
el parche de entrada se realiza una invocación al TCS, resultando en tantas
invocaciones como vértices hay en dicho parche. \\

En el caso de no utilizar un TCS, se pueden pasar valores por defecto a las
siguientes etapas de teselación.

\subsubsection{Generador de primitivas de teselación}
\label{ref:TesPriGen}

El generador de primitivas de teselación~\cite{TesPriGen} es la etapa que se
encuentra entre los dos shaders de teselación, el TCS y el Tessellation
Evaluation Shader. Esta etapa, fija en el pipeline, es la encargada de crear
nuevas primitivas a partir del parche de entrada. La función principal de este
sistema es la de determinar cuántos vértices crear, en qué orden hacerlo y qué
clase de primitivas construir con ellos. Los datos reales de estos vértices,
como color, posición, etc., han de ser generados por el TES. Debido a esto, el
generador no tiene en cuenta el parche de salida producido por el TCS, sino que
solo opera en términos de teselar un cuadrado o triángulo abstracto, o un bloque
de isolíneas.\\

Esta etapa, está supeditada al Tessellation Evaluation Shader, puesto que solo
se ejecutará en el caso de que exista uno activo. La generación de primitivas en
esta etapa se ve afectada por distintos factores:

\begin{itemize}
		\item Niveles de teselación marcados por el TCS (O por defecto si no hay
				TCS)
		\item Espaciado de los vértices teselados, definido en el TES
		\item Tipo de primitiva, definido en el TES
		\item Orden de generación de primitivas, definido en el TES
\end{itemize}

La cantidad de teselación a realizar se define en niveles de teselación internos
y externos. Funcionan de la siguiente manera: un nivel de teselación 4 indica
que un borde se convertirá en 4 bordes (2 vértices se convertirán en 5). El
nivel externo define el grado de teselación para los bordes externos de la
primitiva. Esto permite que dos parches distintos se conecten apropiadamente, a
pesar de tener distintos niveles de teselación dentro del parche. El nivel
interno hace referencia el número de teselaciones a realizar dentro del parche
abstracto. \\

Cabe destacar que no todos los parches abstractos utilizan los mismos niveles de
teselación. Por ejemplo, los triángulos utilizan un único nivel interno y tres
niveles externos. El resto de posibles niveles son ignorados. \\

El espaciado entre vértices puede realizarse de las siguientes maneras:
espaciado equidistante, espaciado fraccional par o espaciado fraccional impar.

\begin{figure}[h]
	\centering
	\begin{subfigure}{.45\textwidth}
			\includegraphics[width=\textwidth]{figures/equal1.png}	
	\end{subfigure}	
	\hfill
	\begin{subfigure}{.45\textwidth}
			\includegraphics[width=\textwidth]{figures/equal2.png}	
	\end{subfigure}	
	\newline
	\begin{subfigure}{.45\textwidth}
			\includegraphics[width=\textwidth]{figures/equal3.png}	
	\end{subfigure}	
	\hfill
	\begin{subfigure}{.45\textwidth}
			\includegraphics[width=\textwidth]{figures/equal4.png}	
	\end{subfigure}	
	\caption{Teselación - Espaciado equidistante}
	\label{fig3.2}
\end{figure}

El espaciado equidistante (ver Figura~\ref{fig3.2}) divide el borde a teselar
en segmentos de igual longitud. Solo acepta valores enteros, por lo que redondea
el nivel de teselación hasta el siguiente entero. Este hecho causa que los
segmentos aparezcan instantáneamente de un nivel a otro.\\

Para conseguir un comportamiento mas ``suave'' se tienen los otros dos modos de
espaciado. Estos últimos son útiles especialmente cuando el nivel de teselación
es dependiente del área vista desde la cámara. En el espaciado fraccional par el
número de segmentos en los que dividir el borde (nivel de teselación efectivo)
se redondea al siguiente entero par, mientras que en el espaciado fraccional
impar se redondea al siguiente entero impar. Para estos modos de espaciado se
necesita definir dos valores:

\begin{itemize}
		\item $n$, el nivel de teselación efectivo, redondeado según lo
				anterior.
		\item $f$, el valor computado antes del redondeo. Un valor
				potencialmente fraccionario.
\end{itemize}

\begin{figure}
	\centering
	\begin{subfigure}{.45\textwidth}
			\includegraphics[width=\textwidth]{figures/even1.png}	
	\end{subfigure}
	\hfill
	\begin{subfigure}{.45\textwidth}
			\includegraphics[width=\textwidth]{figures/even2.png}	
	\end{subfigure}
	\newline
	\begin{subfigure}{.45\textwidth}
			\includegraphics[width=\textwidth]{figures/even3.png}	
	\end{subfigure}
	\hfill
	\begin{subfigure}{.45\textwidth}
			\includegraphics[width=\textwidth]{figures/even4.png}	
	\end{subfigure}
	\caption{Teselación - Espaciado fraccional par}
	\label{fig3.3}
\end{figure}

\begin{figure}
	\centering
	\begin{subfigure}{.45\textwidth}
			\includegraphics[width=\textwidth]{figures/odd1.png}	
	\end{subfigure}
	\hfill
	\begin{subfigure}{.45\textwidth}
			\includegraphics[width=\textwidth]{figures/odd2.png}	
	\end{subfigure}
	\newline
	\begin{subfigure}{.45\textwidth}
			\includegraphics[width=\textwidth]{figures/odd3.png}	
	\end{subfigure}
	\hfill
	\begin{subfigure}{.45\textwidth}
			\includegraphics[width=\textwidth]{figures/odd4.png}	
	\end{subfigure}
	\caption{Teselación - Espaciado fraccional impar}
	\label{fig3.4}
\end{figure}

Según este esquema, los bordes a teselar se subdividen en dos conjuntos de
segmentos. El primero con $n-2$ segmentos de igual longitud, el otro con $2$
segmentos de igual longitud entre ellos, pero no necesariamente de igual
longitud que los del el otro conjunto. Estos dos segmentos tendrán menor
longitud que los otros en general. La longitud de estos es exactamente $n-f$.
Por tanto, cuando se cumple que $n-f=0$ se tiene que todos los segmentos tienen
igual longitud. Estos comportamientos se pueden observar en las
Figuras~\ref{fig3.3},~\ref{fig3.4}.

\subsubsection{Tessellation Evaluation Shader}
\label{ref:TesEvaShader}

El Tessellation Evaluation Shader (TES)~\cite{TesEvaShader} es la etapa opcional
que se encuentra entre el generador de primitivas de teselación y el geometry
shader. Su función es la de coger los resultados obtenidos en la etapa anterior
y computar las posiciones interpoladas y otros datos vértice a vértice a partir
de ellos.\\

El TES obtiene del generador de primitivas de teselación un parche abstracto,
así como datos de los vértices para todo el parche, junto con otros datos,
provenientes del TCS. Cada invocación a este shader produce un vértice
particular y es invocado una vez por cada vértice en el parche abstracto.\\

Esta es la etapa donde el programador implementa el algoritmo que se usa para
computar las nuevas posiciones, normales, coordenadas de texturas, etc. Como se
ha expuesto antes, este shader es el que determina si ocurrirá o no la etapa de
generación de primitivas de teselación, puesto que solo se ejecutará si existe
un TES activo.\\

El TES, en caso de ser utilizado, debe especificar el tipo de primitiva que
servirá como entrada al geometry shader. Este tipo puede ser puntos, isolíneas,
triángulos o cuadriláteros.

\subsection{Geometry Shader}
\label{ref:GeoShader}

El Geometry Shader~\cite{GeoShader} es un programa escrito en GLSL que
corresponde a la etapa del pipeline programable que se encuentra entre el TES o
el Vertex Shader (dependiendo de si existe o no teselación) y la etapa fija de
post-procesado de vértices. Este shader es opcional y no es necesaria su
utilización.\\

Las invocaciones a este shader toman como entrada una única primitiva geométrica
y puede dar como salida cero o más primitivas, aunque existe un límite de
primitivas que se pueden generar en cada invocación, dependiendo de la
implementación. Los shaders geométricos están diseñados para aceptar como
entrada una primitiva específica y dar como salida otra. \\

Sus usos varían bastante, pudiéndose utilizar como una manera de amplificar la
geometría, sirviendo como una especie de teselación, así como para realizar un
renderizado por capas o incluso para la realización de tareas de cómputo en la
GPU. \\

Entre las primitivas de entrada aceptadas por el geometry shader se encuentran
las siguientes:

\begin{table}[h]
		%\centering
		\begin{tabular}{|m{4cm}|m{7cm}|m{2.2cm}|m{1.5cm}|}

			\hline
			Entrada & Primitiva & Parámetro TES & Vértices\\
			\hline

			\verb|points| & \verb|GL_POINTS| & \verb|point_mode| & 1 \\
			\hline

			\verb|lines| & \verb|GL_LINES, GL_LINE_STRIP,| \verb|GL_LINE_LIST| &
			\verb|isolines| & 2\\

			\hline

			\verb|lines_adjacency| & \verb|GL_LINES_ADJACENCY,|
			\verb|GL_LINE_STRIP_ADJACENCY| & \verb|N/A| & 4\\

			\hline

			\verb|triangles| & \verb|GL_TRIANGLES, GL_TRIANGLE_STRIP,|
			\verb|GL_TRIANGLE_FAN| & \verb|triangles,| \verb|quads| & 3 \\

			\hline

			\verb|triangles_adjacency| & \verb|GL_TRIANGLES_ADJACENCY,|
			\verb|GL_TRIANGLE_STRIP_ADJACENCY| & \verb|N/A| & 6 \\

			\hline
		\end{tabular}
		\caption{Primitivas de entrada al Geometry Shader}
		\label{tabla3.1}
\end{table}

Las primitivas de salida pueden ser únicamente alguna de las siguientes:

\begin{itemize}
		\item \verb|points|	
		\item \verb|line_strip|
		\item \verb|triangle_strip|
\end{itemize}

Los shaders geométricos pueden generar tantos vértices como permita el límite de
implementación. Para ello, el programador genera los valores que necesite para
el nuevo vértice y, una vez estos valores sean correctos, una llamada a la
función \verb|EmitVertex()| produce el vértice deseado. Una vez llamada esta
función, los valores escritos para el vértice son reseteados, teniendo que
volver a escribirlos para generar otro vértice. \\

De igual modo, para generar una primitiva, debemos especificar del modo anterior
todos los vértices que forman esa primitiva y posteriormente llamar a la función
\verb|EndPrimitive()|. De esta forma, si se desea generar más de una primitiva,
se deben especificar los vértices que forman la primera, llamar a
\verb|EndPrimitive()|, generar los vértices que forman la segunda y llamar de
nuevo a \verb|EndPrimitive()| para generar la segunda primitiva.

\subsection{Fragment Shader}
\label{ref:FragShader}

El Fragment Shader~\cite{FragShader} es la etapa posterior a la rasterización.
Por cada uno de los píxeles cubiertos por una primitiva, se genera un fragmento.
Cada uno de estos fragmentos tiene una posición en la espacio de ventana, así
como otros valores procedentes de la etapa de procesamiento de vértices. \\ 

La salida del fragment shader consta de un valor de profundidad, un posible
valor de plantilla (que no es modificado por el shader) y cero o más valores de
color para ser potencialmente escritos en los buffers del frame buffer actual.
Estos shaders toman como entrada un único fragmento, producido por el
rasterizador, y dan como salida otro único fragmento. \\

Técnicamente, la utilización de estos shaders es también opcional, puesto que de
no utilizarlo, los valores de color del fragmento de entrada quedarán
indefinidos, pero los valores de profundidad y plantilla en la salida serán los
mismos que los de entrada. Esto puede ser interesante en el caso de solo estar
interesados en los valores de profundidad computados por el sistema en lugar de
otro valor calculado por el programador. \\

Este shader también tiene operaciones especiales no presentes en los otros tipos
de shader, como puede ser la instrucción \verb|discard|, cuyo objetivo es
descartar los valores de salida generados durante la ejecución del shader para
un fragmento en concreto, haciendo que este fragmento no pase a las siguientes
etapas del pipeline. Esto puede ser útil para descartar fragmentos cuyos valores
generados en la ejecución se queden fuera de unos límites impuestos por el
programador.

\section{Uso en Visualización Científica}
\label{ref:SciVis}

Una vez entendido cómo funcionan los shaders y qué entradas y salidas toman, en
esta sección se explora sus posibles aplicaciones en la disciplina de la
visualización científica. Para ello, mostraremos algunos problemas típicos de
visualización y analizaremos cómo resolver estos problemas gracias a las
capacidades que cada uno de los shaders nos proporcionan.

\cleardoublepage

\chapter{Aplicación}
\label{makereference4}

En este capítulo se presenta la aplicación que se ha desarrollado para ilustrar
y poner en práctica las ideas aprendidas.

\cleardoublepage

\chapter{Aplicación desarrollada}
\label{makereference5}

En este capítulo se introducirá la aplicación que se ha desarrollado con el
fin de demostrar los conceptos expuestos en los capítulos anteriores. En
concreto se desarrollarán shaders para los siguientes problemas:

\begin{itemize}
		\item Coloreado de terrenos
		\item Curvas de Bézier
		\item Superficies de Bézier
		\item Sólidos de revolución
		\item Nubes de puntos
		\item Negativo de una imagen
		\item Detección de bordes en una imagen
		\item Line Integral Convolution
\end{itemize}

Para el desarrollo de la aplicación y los shaders será necesario introducir
algunos conceptos matemáticos importantes, que se explican en la
sección~\ref{makereference5.1}.

\section{Plan de desarrollo}
\label{makereference5.1}

Durante las primera semanas de desarrollo de la aplicación lo más importante fue
realizar un exhaustivo estudio del funcionamiento de OpenGL, así como la lectura
y realización de tutoriales sobre la materia. Una vez adquirido el conocimiento
necesario, se comenzó a desarrollar el esqueleto principal de la aplicación,
sobre el cual se incorporarían después los distintos tipos de visualización a
realizar.\\

Una vez desarrollado este esqueleto se comenzó a la preparación de los shaders
que se utilizarían en cada uno de los problemas, para lo que se utilizó en gran
medida la información expuesta en el texto~\citet{Bailey}.\\

El siguiente paso fue conseguir datos y prepararlos adecuadamente para poder
mostrar las capacidades de visualización de la aplicación, así como comprobar su
correcto funcionamiento.\\

Lo anterior fue reiterado con cada uno de los problemas, añadiéndose cada vez
más a lo largo del desarrollo de todo el proyecto.

\section{Herramientas de desarrollo}
\label{makereference5.2}

Como entorno de desarrollo principal se ha utilizado el sistema operativo Ubuntu
Linux 18.04 LTS~\cite{UBUNTU}. En este sistema, además, se han utilizado las
siguientes herramientas de desarrollo:

\begin{itemize}
		\item Vim como editor de textos.~\cite{VIM}
		\item Git para el control de versiones.~\cite{GIT}
		\item Github como respositorio.~\cite{GITHUB}
		\item GCC como compilador.~\cite{GCC}
		\item GDB para la depuración.~\cite{GDB}
		\item Make para la gestión de dependencias.~\cite{MAKE}
\end{itemize}. 

Asimismo, siguiendo las recomendaciones del tutorial~\citet{LearnOpenGL}, se han
utilizado las siguientes librerías:

\begin{itemize}
		\item GLFW~\cite{GLFW}. GLFW es una librería orientada específicamente a
				OpenGL que proporciona las necesidades básicas para el
				renderizado en pantalla. Permite crear un contexto de OpenGL,
				definir parámetros de ventana y manejar la entrada del usuario.
				Estas son las funciones que utilizaremos en la aplicación.
		\item GLAD~\cite{GLAD}. La localización de funciones de OpenGL depende
				tanto del controlador gráfico utilizado como del sistema
				operativo utilizado. Esta localización es desconocida en tiempo
				de  compilación y ha de ser conseguida en tiempo de ejecución.
				Es, pues, tarea del programador conseguir la localización de
				estas funciones. GLAD es una  librería que realiza esta tarea
				automáticamente.
		\item Assimp~\cite{ASSIMP}. Assimp --- \textit{The
				Open-Asset-Importer-Lib} --- es una librería que permite
				importar diferentes formatos de modelos 3D de una manera
				uniforme. Será utilizado para cargar los modelos para el
				colorado de terrenos.
		\item GLM~\cite{GLM}. GLM --- OpenGL Mathematics --- es una librería
				para matemáticas en software gráfico en C++ basada en las
				especificaciones del lenguaje GLSL. Proporciona funciones
				diseñadas e implementadas con el mismo convenio de nombres y
				funcionalidades que GLSL. Proporciona capacidades como
				transformaciones de matrices, cuaterniones, empaquetado de
				datos, aleatoriedad, ruido\ldots
		\item Otras librerías especificas del sistema operativo, como Pthreads,
				xrandr, x11, xi, xcursor, etc.
\end{itemize}

\section{Diseño de la aplicación}
\label{makereference5.3}

Para esta aplicación se ha optado por un diseño modular orientado a objetos, en
el que poder incrementalmente añadir distintos tipos de visualización sin tener
demasiados problemas. Como se ha expuesto en la sección~\ref{makereference5.1}
la aplicación consta de un esqueleto principal utilizado por todos los tipos de
visualización. Este esqueleto consta de una ventana principal, creada en el
programa principal, en la que se renderizará el objeto particular que representa
el tipo de visualización. Así, con el fin de añadir un nuevo tipo de
visualización solo se habrían de realizar las siguientes acciones:

\begin{enumerate}
		\item Crear un nuevo objeto que implemente los métodos necesarios
				de la clase \verb|Object|.
		\item Añadir un nuevo modo al \verb|enum Modes|.
		\item Añadir las opciones necesarias para dicho objeto en el programa
				principal e incluir la nueva clase \verb|#include "class.h"|.
		\item Actualizar las dependencias en el Makefile.
\end{enumerate}

En esta ventana principal se tiene por defecto un sistema de cámara en primera
persona, con la capacidad de moverse para visualizar mejor detalles del objeto
en cuestión. Este sistema puede ser sobreescrito en el objeto específico en
caso de necesitar otro comportamiento. \\

En la clase \verb|Object| existen tres métodos virtuales puros, que han de ser
implementados por las clases específicas de cada tipo de visualización:

\begin{itemize}
		\item \verb|draw()|, que ha de encargarse de dibujar el objeto en
				cuestión.
		\item \verb|processInputGLFWwindow * window)|, que ha de especificar que
				hacer con la entrada del usuario para este tipo de
				visualización.
		\item \verb|setUniforms()|, que ha de especificar las variables
				\verb|uniform| que utilicen los shaders de este tipo de
				visualización.
\end{itemize}

Estos métodos se llaman una vez por vuelta del bucle principal. En la sección
siguiente se introducen las matemáticas necesarias e importantes para el
desarrollo de la aplicación y los shaders concretos.

\section{Matemáticas necesarias}
\label{makereference5.4}

Con el fin de desarrollar el sistema de cámaras que utiliza la aplicación es
necesario conocer varios conceptos importantes sobre álgebra, geometría lineal y
transformaciones matriciales, así como los ángulos de Euler y su relación con
los cuaterniones. También se explorarán distintos métodos numéricos, relevantes
en el método de Line Integral Convolution, así como fórmulas matemáticas
específicas de cada tipo de visualización.

\subsection{Transformaciones matriciales}
\label{makereference5.4.1}

Como vamos a trabajar con objetos tridimensionales y una cámara móvil,
necesitamos realizar transformaciones sobre los vértices que componen nuestros
objetos para que estos aparezcan en su lugar y con sus dimensiones adecuadas.
Es deseable, además, realizar estas operaciones con vectores matricialmente,
puesto que éstas permiten presentar transformaciones arbitrarias en un
formato consistente y apto para la computación. Así, se pueden concatenar
diferentes transformaciones de manera sencilla multiplicando sus matrices. \\

Entre estas transformaciones podemos encontrar lineales y no lineales. Así, para
representar matricialmente transformaciones no lineales en un espacio Euclídeo
$n$-dimensional $\mathbb{R}^n$ se puede utilizar una transformación lineal en el
espacio $(n+1)$-dimensional $\mathbb{R}^{n+1}$. Este tipo de transformaciones
incluye tanto las transformaciones afines (como la
traslación~\ref{makereference5.4.1.1}) como transformaciones proyectivas.\\

Esta es la razón por la que las matrices $4 \times 4$ son tan ampliamente
utilizadas en la informática gráfica y, en consecuencia, en nuestra
aplicación.\\

En esta sección se presentan las transformaciones lineales y afines más
habituales y necesarias para nuestra aplicación, así como sus formas
matriciales. \\

\subsubsection{Traslación}
\label{makereference5.4.1.1}

Se denomina traslación a la operación consistente en \textit{mover} un vector en
una posición a otra nueva posición. Supongamos, pues, que queremos trasladar un
vector $\overrightarrow{v} = (x,y,z)$ en la dirección marcada por el vector
$\overrightarrow{t} = (t_1, t_2, t_3)$ como se muestra en la
figura~\ref{fig:translation}. Para ello realizaríamos la siguiente operación:\\

\begin{equation}
	\label{eq:translation}
	\overrightarrow{v}' = \overrightarrow{v} + \overrightarrow{t} = 
	\left( \begin{array}{c}
			x + t_1 \\
			y + t_2 \\
			z + t_3 \\
	\end{array} \right)
\end{equation}\\

La traslación se trata de una transformación afín sin puntos fijos. Como se ha
expuesto previamente, para poder representar esta transformación de forma
matricial se ha de recurrir a un espacio de una dimensión más. Por tanto, se
recurre a las coordenadas homogéneas para representar la traslación de un
espacio vectorial con multiplicación de matrices. Escribiendo el vector
$\overrightarrow{v} = (x,y,z)$ utilizando una cuarta coordenada homogénea
$\overrightarrow{v} = (x,y,z,1)$. Esta operación se muestra
en~\eqref{eq:matrixtranslation}. \\

\begin{equation}
	\label{eq:matrixtranslation}
	\overrightarrow{v}' = 
	\left( \begin{array}{cccc}
			1 & 0 & 0 & t_1 \\
			0 & 1 & 0 & t_2 \\
			0 & 0 & 1 & t_3 \\
			0 & 0 & 0 & 1 \\
	\end{array} \right)
	\left( \begin{array}{c}
			x \\
			y \\
			z \\
			1 \\
	\end{array} \right) = 
	\left( \begin{array}{c}
			x + t_1 \\
			y + t_2 \\
			z + t_3 \\
			1 \\
	\end{array} \right)
\end{equation}\\

\subsubsection{Escalado}
\label{makereference5.4.1.2}

El escalado de un vector es la operación consistente en modificar la longitud
del vector. Para ello, debemos multiplicar cada una de sus coordenadas por el
factor de escalado deseado en cada eje. Es decir, para escalar un vector
$\overrightarrow{v}$ por un factor de $0.5$ en el eje $x$ y $3$ en el eje $y$,
la operación a realizar sería la siguiente:\\

\begin{equation}
	\label{eq:scaling}
	\overrightarrow{v} = 
	\left( \begin{array}{c}
			2 \\
			3 \\
	\end{array} \right)
	\;\;\;\;\;
	\overrightarrow{v}' =
	\left( \begin{array}{c}
		2\cdot0.5 \\
		3\cdot3 \\
	\end{array}	\right) =  
	\left( \begin{array}{c}
			1 \\
			9 \\
	\end{array} \right)
\end{equation}\\

\begin{figure}[h]
	\centering
	\includegraphics{figures/scaling.png}
	\caption{Escalar un vector}
	\label{fig:scaling}
\end{figure}

Esta operación de escalado se puede escribir matricialmente como sigue.
Supongamos que tenemos un vector $\overrightarrow{v}=(x,y,z)$ y lo queremos
escalar por un factor $fac=(F_1,F_2,F_3)$. Entonces podemos escribir la
operación anterior con la matriz $FAC$ como sigue:\\

\begin{equation}
	\label{eq:matrixscaling}
	\overrightarrow{v}' = 
	\left( \begin{array}{cccc}
			F_1 & 0   & 0   & 0	\\
			0   & F_2 & 0   & 0	\\
			0   & 0   & F_3 & 0	\\
			0   & 0   & 0   & 1	\\
	\end{array} \right)
	\left( \begin{array}{c}
			x \\
			y \\
			z \\
			1 \\
	\end{array} \right) =
	\left( \begin{array}{c}
			F_1\cdot x \\
			F_2\cdot y \\
			F_3\cdot z \\
			1 \\
	\end{array} \right)
\end{equation}\\

Nótese que también en este caso al vector $\overrightarrow{v}$ se le ha añadido
una cuarta coordenada $w$ con el fin de ser consistentes con aquellas
transformaciones no lineales que necesitan de un espacio de dimensión mayor. \\

\subsubsection{Rotación}
\label{makereference5.4.1.3}

Al contrario de los casos de la rotación y traslación de vectores expuestas
anteriormente, el caso de la rotación requiere un estudio más profundo en el
caso tridimensional para la matemática aplicada. Por esto, se dedica una sección
exclusiva para este tema. (Ver sección~\ref{makereference5.4.2}).

\subsection{Rotación: Ángulos de Euler y Cuaterniones}
\label{makereference5.4.2}

%% TODO

\subsection{Métodos Numéricos para la resolución de Ecuaciones Diferenciales}
\label{makereference5.4.3}

Como ya se vio en la sección~\ref{ref:lic}, en el método del Line Integral Convolution
se ha de utilizar un método numérico para computar la línea de flujo. En esta
sección se analizan algunos de los métodos utilizados y propuestos tanto
por~\citet{osti_10185520} como por \citet{licthesis}. 

\subsubsection{Método de Euler}
\label{makereference5.4.3.1}
\subsubsection{Método de Euler de paso variable}
\label{makereference5.4.3.2}
\subsubsection{Métodos de Runge-Kutta}
\label{makereference5.4.3.3}

\subsection{Otras Fórmulas}
\label{makereference5.4.4}

Para el desarrollo de la aplicación también se han tenido que utilizar fórmulas
matemáticas que describen el comportamiento de las curvas y superficies de
Bézier, ya explicadas en la sección~\ref{ref:bezier}, así como las
fórmulas para la obtención de los valores de los pixeles de salida en el método
del Line Integral Convolution (Sección~\ref{ref:lic}).

\section{Shaders en la aplicación}
\label{makereference5.5}

En esta sección se presentan ya los shaders desarrollados para cada uno de los
tipos de visualización considerados en el proyecto, analizando los distintos
cálculos realizados y explorando las entradas y salidas de cada uno de ellos.

\subsection{Coloreado de terrenos}
\label{makereference5.5.1}
\subsection{Curvas de Bézier}
\label{makereference5.5.2}
\subsection{Superficies de Bézier}
\label{makereference5.5.3}
\subsection{Sólidos de revolución}
\label{makereference5.5.4}
\subsection{Nube de puntos}
\label{makereference5.5.5}
\subsection{Negativo de una imagen}
\label{makereference5.5.6}
\subsection{Detección de bordes}
\label{makereference5.5.7}
\subsection{Line Integral Convolution}
\label{makereference5.5.8}


% +-------------------------------------------------------------------------+
% | References                                                              |
% +-------------------------------------------------------------------------+

% +-------------------------------------------------------------------------+
% | In order for WinEDT to index references correctly, it has to know where |
% | the file resides.  The following command is prefaced by %, and will be  |
% | ignored completely by LaTeX.  However, WinEDT will use this line to     |
% | access the external .bib bibliography file.  Also note that WinEDT can  |
% | read file path names with either "\" or "/" - LaTeX, however, doesn't   |
% | like "\", so it's easier to store a path name in the "Unix" style.      |
% +-------------------------------------------------------------------------+

%Included for Gather Purpose only.  Do NOT uncomment:
%input "references.bib"

% +--------------------------------------------------------------------+
% | This template uses the BibTeX program to format references.  The
% | 3 lines below create a separate Bibliography section and add
% | an entry for "Bibliography" to the Table of Contents.  The actual
% | data for your references (author, title, journal, date, etc.) are
% | entered in the references.bib file.  See that file for information
% | on how to enter references.
% +--------------------------------------------------------------------+

\bibdata{references}
\bibliography{references}
\addcontentsline{toc}{chapter}{Bibliography}

% +--------------------------------------------------------------------+
% | Finally, we generate the appendix.  To add or delete appendices,
% | add or remove the line
% |
% |     \input{appendixX.tex}
% |
% | where "X" is the letter designation of the Appendix (A, B, C, etc.)
% | You should have one \input{appendixX.tex} line and a corresponding
% | file appendixX.tex for each appendix.                                 |
% +--------------------------------------------------------------------+

\appendix
% +--------------------------------------------------------------------+
% | Appendix A Page (Optional)                                         |
% +--------------------------------------------------------------------+

\cleardoublepage

\chapter{El lenguaje GLSL}
\label{ApendiceA}

En este apéndice se tratan algunas de las características del lenguaje GLSL,
sobre todo aquellas necesarias para la implementación de los shaders vistos en
la sección~\ref{makereference5.5}. Para una guía completa sobre el lenguaje se
puede consultar la documentación detallada~\cite{GLSLreference}.

\section{Versión}

Un programa GLSL ha de comenzar obligatoriamente con una declaración de la
versión de GLSL que se va a utilizar. Esto se realiza mediante una directiva que
se utiliza para saber la versión que se ha de utilizar para compilar el
programa. Esta directiva se declara del siguiente modo:

\begin{verbatim}
    #version 400
\end{verbatim}
En este caso se ha de utilizar la versión GLSL 4.0.

\section{Funciones y Estructuras de Control}

Como estructuras de control, GLSL soporta bucles y saltos, con las instrucciones
comunes del lenguaje C (if-else, switch, while, for\ldots). Sin embargo, la
recursión en este lenguaje no está permitida. 

En cuanto a las funciones, este lenguaje soporta tanto funciones definidas por
el usuario como funciones proporcionadas por el propio lenguaje. Una lista de
las funciones incluidas puede encontrarse en~\citet{GLSLreference}. De entre
estas funciones, se han utilizado las siguientes en los shaders desarrollados:

\begin{itemize}
		\item \verb|smoothstep| - Realiza una interpolación de Hermite entre dos
				valores~\cite{Hermite}.
		\item \verb|mix| - Realiza una interpolación lineal entre dos
				valores~\cite{LinearInterpolation}.
		\item \verb|normalize| - Calcula el vector unitario correspondiente a la
				dirección del vector original.
		\item \verb|max| - Devuelve el máximo entre dos valores.
		\item \verb|dot| - Calcula el producto escalar entre dos vectores.
		\item \verb|reflect| - Calcula la dirección de reflexión del vector
				incidente.
		\item \verb|pow| - Devuelve la potencia del primer parámetro elevado al
				segundo parámetro.
		\item \verb|EmitVertex| - Explicado en la sección~\ref{ref:GeoShader}.
		\item \verb|sin| - Devuelve el seno del parámetro.
		\item \verb|cos| - Devuelve el coseno del parámetro.
		\item \verb|texture| - Devuelve texels de una textura.
		\item \verb|clamp| - Restringir un valor para que se encuentre entre
				otros dos valores.
		\item \verb|discard| - Descarta el fragmento actual (Solo para el
				fragment shader).
\end{itemize}

\section{Tipos de Datos}

GLSL define una serie de tipos de datos. Algunos de estos son compartidos con
los lenguajes C y C++, mientras que otros son completamente nuevos. En nuestro
caso, los utilizados son los siguientes:

\subsection{Escalares}

\begin{itemize}
		\item \verb|bool| - Condicional. Valores \verb|true| o \verb|false|.
		\item \verb|int| - Entero con signo de 32 bits.
		\item \verb|float| - Número de coma flotante.
\end{itemize}

\subsection{Vectoriales}

GLSL soporta tipos de datos vectoriales de $n$ componentes, con $n$ siendo $2,
3$ o $4$. Los tipos vectoriales soportan las mismas operaciones que los
escalares. Estas operaciones se realizan componente a componente. Sin embargo,
estas operaciones funcionarán solo si los dos vectores tienen el mismo número de
componentes.

\begin{itemize}
		\item \verb|vec|$n$ - Un vector de $n$ componentes tipo
				\verb|float|.
\end{itemize}

\subsection{Swizzling}

El swizzling consiste en la posibilidad que ofrece GLSL de acceder a las
componentes de un vector de la siguiente manera:

\begin{verbatim}
    vec4 someVec;
    someVec.x + someVec.y;
\end{verbatim}

Se pueden utilizar \verb|x|, \verb|y|, \verb|z| y \verb|w| para acceder a las
componentes primera, segunda, tercera y cuarta respectivamente. Además se puede
construir un nuevo vector a partir de otro especificando el orden de las
coordenadas del vector antiguo en el nuevo. Con todo esto, algunos ejemplos más
ilustrativos del swizzling son los siguientes:

\begin{verbatim}
    vec2 someVec;
    vec4 otherVec = someVec.xyxx;
    vec3 thirdVec = otherVec.zyy;	
    vec4 someVec;
    someVec.wzyx = vec4(1.0, 2.0, 3.0, 4.0); // Da la vuelta al vector.
    someVec.zx = vec2(3.0, 5.0); // Pone la componente z a 3 y la x a 5.
\end{verbatim}

Además, para facilitar el acceso a un vector que represente un color o a uno que
represente coordenadas  de una textura, se pueden utilizar, juntos a las letras
anteriores, las siguientes combinaciones:

\begin{itemize}
		\item \verb|rgba|
		\item \verb|stpq|
\end{itemize}

Estas combinaciones no suponen ninguna diferencia con la principal, son
simplemente azúcar sintáctico para ayudar al programador. Sin embargo, no se
pueden realizar combinaciones que contengan letras de distintos tipos. Por
ejemplo, la combinación: 
\begin{verbatim}
    vec3 aVec = someVec.xbg;
\end{verbatim}
resultaría en un error de compilación.

\subsection{Matrices}

También soporta tipos matriciales. Para declarar una matriz se utiliza la
siguiente sintaxis:

\begin{itemize}
		\item \verb|mat|$n$\verb|x|$m$ - Matriz $n \times m$
		\item \verb|mat|$n$ - Matriz $n \times n$
\end{itemize}	
donde $n$ puede ser $2, 3$ o $4$. En los tipos matriciales no se puede utilizar
swizzling, por lo que han de ser accedidos mediante una sintaxis de array propia
de C o C++.

\subsection{Uniforms}

Los \verb|uniform| son un tipo de variable de shader. Se utilizan como
parámetros que el programador de un shader puede pasar al programa principal. Se
llaman de esta manera porque su valor no cambia de una invocación de shader a
otra durante la misma llamada de renderizado. 

Estas variables han de ser declaradas de un modo global dentro del shader y
pueden ser de cualquiera  de los tipos permitidos en GALS. Son variables  de
solo lectura dentro del shader, y cualquier intento de cambio de su valor
resultará en un error de compilación. Están pensados para utilizarse en el
programa desde OpenGL y no desde el shader.

Para utilizar una variable \verb|uniform| desde la aplicación, se ha de recabar
la localización del shader mediante una llamada a la función
\verb|glGetUniformLocation|.

\section{Variables y Entrada/Salida}

Los shaders han de ser capaces de comunicarse entre si dentro del pipeline. Las
entradas y salidas de cada uno de los tipos de shader se han explicado en la
sección~\ref{makereference3}. En esta sección se expone la sintaxis necesarias
para declarar estas variables de entrada y salida desde cada tipo de shader.

Asimismo, existe una serie de variables específicas construidas dentro del
lenguaje GLSL que tienen un significado ya definido.

\subsection{Entradas y salidas}

De manera general, una variable de entrada a un shader ha  de llevar el
modificador \verb|in|. De igual modo, una variable de salida dentro de un shader
ha de llevar el modificador \verb|out|. Por tanto, si queremos pasar un valor
desde el vertex shader al fragment shader (suponiendo que no se utilizan
tessellation ni geometry shaders) declararíamos lo siguiente en el vertex shader:

\begin{verbatim}
    out vec3 vPos;	
\end{verbatim}

Y, de manera correspondiente, lo siguiente en el fragment shader:

\begin{verbatim}
    in vec3 vPos;	
\end{verbatim}

Un caso particular proviene de las entradas del vertex shader. Al ser el primero
en el pipeline, las entradas a este shader provienen de los datos contenidos en
un VBO. Estos datos corresponden a los diferentes atributos de un vértice en
particular, y están referenciados en un VAO. Así, el primer atributo de un
vértice, especificado en el VAO, puede ser la posición, mientras que el segundo
puede ser el color. Para pasar esta información al shader se declaran las
siguientes líneas dentro del shader:

\begin{verbatim}
    layout(location = 1) in vec3 aPos;		
    layout(location = 2) in vec3 aColor;		
\end{verbatim}

Esto quiere decir al shader que en el primer atributo ha de encontrar
un tipo \verb|vec3| que representa la posición y en el segundo atributo ha de
encontrar otro \verb|vec3| que representa el color del vértice correspondiente.

De igual modo se pueden especificar salidas desde la última etapa del pipeline,
el fragment shader.

Además de estos dos casos particulares, se dan los de aquellos shaders que
necesitan instrucciones específicas para realizar su función. Son los
siguientes:

\subsubsection{TCS}

Para utilizar el shader de teselación, se ha de habilitar una extensión ARB, que
se realiza mediante la siguiente directiva.

\begin{verbatim}
    #extension GL_ARB_tessellation_shader : enable
\end{verbatim}

Además el TCS, acorde a lo expuesto en la seccion~\ref{ref:TesConShader}, ha de
especificar el número de vértices en el parche de salida. Esto no se realiza
mediante variables específicas, sino mediante la siguiente instrucción:

\begin{verbatim}
    layout( vertices = 16 ) out;
\end{verbatim}

\subsubsection{TES}

En este caso, al igual que en el TCS, se ha de habilitar la extensión ARB que
permite la utilización de tessellation shaders, por lo que se ha de incluir la
misma línea. 

Además, en este caso se ha de especificar lo explicado en la
sección~\ref{ref:TesEvaShader} (espaciado, primitiva, orden), que se realiza
mediante la siguiente instrucción:

\begin{verbatim}
    layout( quads, equal_spacing, ccw ) in;
\end{verbatim}

\subsubsection{Geometry shader}

Para habilitar la utilización de este shader se ha de habilitar otra extensión
ARB:

\begin{verbatim}
    #extension GL_EXT_geometry_shader4: enable
\end{verbatim}

Además se ha de definir tanto la primitiva de entrada como la de salida y el
número máximo de vértices que se pueden generar (Sección~\ref{ref:GeoShader}).
Para ello hay que utilizar las siguientes líneas:

\begin{verbatim}
    layout( lines_adjacency ) in;
    layout( line_strip, max_vertices=256 ) out;
\end{verbatim}

\subsection{Variables Específicas}

Además de las variables definidas por el usuario y de las instrucciones
específicas necesarias anteriores, GLSL tiene algunas variables ya definidas que
se utilizan para propósitos específicos, aunque no es necesario utilizarlas.
Estas variables pueden ser específicas para cada tipo de shader o comunes a
todos ellos. Una lista de estas variables puede encontrarse
en~\citet{GLSLreference}. Aquí se muestran las utilizadas en la aplicación.

\subsubsection{Vertex Shader}

\begin{itemize}
		\item \verb|out vec4 gl_Position| - La posición del vértice actual en el
				clip space.
		\item \verb|out float gl_PointSize| - El tamaño del punto siendo
				rasterizado. Solo funciona con la primitiva \verb|GL_POINTS|.
\end{itemize}

\subsubsection{TCS}

\begin{itemize}
		\item \verb|in int gl_InvocationID| - El índice de la invocación del TCS
				dentro del patch. El TCS escribe a las variables de salida
				vértice a vértice utilizando esta variable para indexarlos.
		\item \verb|patch out float gl_TessLevelOuter[4]| - Niveles externos de
				teselación.
		\item \verb|patch out float gl_TessLevelInner[2]| - Niveles internos de
				teselación.
		\item \verb|gl_in[gl_MaxPatchVertices]| - Información de cada vértice
				proveniente del vertex shader.
\end{itemize}

\subsubsection{TES}

\begin{itemize}
		\item \verb|in vec3 gl_TessCoord| - La localización dentro del abstract
				patch para cada vértice particular.
		\item \verb|gl_in[gl_MaxPatchVertices]| - Información de cada vértice
				proveniente del TCS.
\end{itemize}

\subsubsection{Geometry Shader}

\begin{itemize}
		\item \verb|gl_Position| - La posición del vértice actual dentro del
				clip space.
\end{itemize}

% \input{appendixB.tex}

\end{document}
