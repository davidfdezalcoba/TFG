% +--------------------------------------------------------------------+
% | Copyright Page
% +--------------------------------------------------------------------+

\newpage

\thispagestyle{empty}

\begin{center}

{\bf \Huge Abstract}

  \end{center}
\vspace{1cm}

	Nowadays, scientific studies and investigations generate a great amount of
	data that has to be well interpreted, so as to extract the best possible
	conclusions and obtain reliable results. Within this context, one of the
	most important disciplines is that of visualization, since it can help
	understand, illustrate and obtain relevant information about the phenomenon
	being studied.


	Similarly, in the last few years, a considerable expansion of the
	capabilities of GPUs has been taking place, offering new possibilities
	within graphics computing and increasing performance both in parallel
	computing and in visualization applications.


	In this text those ideas are explored, emphasizing the possibilities that
	the different kinds of graphic shaders  have to offer within the OpenGL
	specification, and the way these can help interpret data and obtain
	representations for common scientific visualization topics, such as three
	dimensions data Visualization, volume visualization, surface rendering, etc.

\vspace{1cm}

% +--------------------------------------------------------------------+
% | On the line below, replace Fecha
% |
% +--------------------------------------------------------------------+

\begin{center}

{\bf \Large Keywords}

   \end{center}

   \vspace{0.5cm}
   
	Scientific Visualization, Graphic Shaders, GPU, OpenGL.
   


