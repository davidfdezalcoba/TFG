\cleardoublepage

\chapter{Aplicación desarrollada}
\label{makereference5}

En este capítulo se introducirá la aplicación que se ha desarrollado con el
fin de demostrar los conceptos expuestos en los capítulos anteriores. En
concreto se desarrollarán shaders para los siguientes problemas:

\begin{itemize}
		\item Coloreado de terrenos
		\item Curvas de Bézier
		\item Superficies de Bézier
		\item Sólidos de revolución
		\item Nubes de puntos
		\item Negativo de una imagen
		\item Detección de bordes en una imagen
		\item Line Integral Convolution
\end{itemize}

Para el desarrollo de la aplicación y los shaders será necesario introducir
algunos conceptos matemáticos importantes, que se explican en la
sección~\ref{makereference5.1}.

\section{Plan de desarrollo}
\label{makereference5.1}

Durante las primera semanas de desarrollo de la aplicación lo más importante fue
realizar un exhaustivo estudio del funcionamiento de OpenGL, así como la lectura
y realización de tutoriales sobre la materia. Una vez adquirido el conocimiento
necesario, se comenzó a desarrollar el esqueleto principal de la aplicación,
sobre el cual se incorporarían después los distintos tipos de visualización a
realizar.\\

Una vez desarrollado este esqueleto se comenzó a la preparación de los shaders
que se utilizarían en cada uno de los problemas, para lo que se utilizó en gran
medida la información expuesta en el texto~\citet{Bailey}.\\

El siguiente paso fue conseguir datos y prepararlos adecuadamente para poder
mostrar las capacidades de visualización de la aplicación, así como comprobar su
correcto funcionamiento.\\

Lo anterior fue reiterado con cada uno de los problemas, añadiéndose cada vez
más a lo largo del desarrollo de todo el proyecto.

\section{Herramientas de desarrollo}
\label{makereference5.2}

Como entorno de desarrollo principal se ha utilizado el sistema operativo Ubuntu
Linux 18.04 LTS~\cite{UBUNTU}. En este sistema, además, se han utilizado las
siguientes herramientas de desarrollo:

\begin{itemize}
		\item Vim como editor de textos.~\cite{VIM}
		\item Git para el control de versiones.~\cite{GIT}
		\item Github como respositorio.~\cite{GITHUB}
		\item GCC como compilador.~\cite{GCC}
		\item GDB para la depuración.~\cite{GDB}
		\item Make para la gestión de dependencias.~\cite{MAKE}
\end{itemize}. 

Asimismo, siguiendo las recomendaciones del tutorial~\citet{LearnOpenGL}, se han
utilizado las siguientes librerías:

\begin{itemize}
		\item GLFW~\cite{GLFW}. GLFW es una librería orientada específicamente a
				OpenGL que proporciona las necesidades básicas para el
				renderizado en pantalla. Permite crear un contexto de OpenGL,
				definir parámetros de ventana y manejar la entrada del usuario.
				Estas son las funciones que utilizaremos en la aplicación.
		\item GLAD~\cite{GLAD}. La localización de funciones de OpenGL depende
				tanto del controlador gráfico utilizado como del sistema
				operativo utilizado. Esta localización es desconocida en tiempo
				de  compilación y ha de ser conseguida en tiempo de ejecución.
				Es, pues, tarea del programador conseguir la localización de
				estas funciones. GLAD es una  librería que realiza esta tarea
				automáticamente.
		\item Assimp~\cite{ASSIMP}. Assimp --- \textit{The
				Open-Asset-Importer-Lib} --- es una librería que permite
				importar diferentes formatos de modelos 3D de una manera
				uniforme. Será utilizado para cargar los modelos para el
				colorado de terrenos.
		\item GLM~\cite{GLM}. GLM --- OpenGL Mathematics --- es una librería
				para matemáticas en software gráfico en C++ basada en las
				especificaciones del lenguaje GLSL. Proporciona funciones
				diseñadas e implementadas con el mismo convenio de nombres y
				funcionalidades que GLSL. Proporciona capacidades como
				transformaciones de matrices, cuaterniones, empaquetado de
				datos, aleatoriedad, ruido\ldots
		\item Otras librerías especificas del sistema operativo, como Pthreads,
				xrandr, x11, xi, xcursor, etc.
\end{itemize}

\section{Diseño de la aplicación}
\label{makereference5.3}

Para esta aplicación se ha optado por un diseño modular orientado a objetos, en
el que poder incrementalmente añadir distintos tipos de visualización sin tener
demasiados problemas. Como se ha expuesto en la sección~\ref{makereference5.1}
la aplicación consta de un esqueleto principal utilizado por todos los tipos de
visualización. Este esqueleto consta de una ventana principal, creada en el
programa principal, en la que se renderizará el objeto particular que representa
el tipo de visualización. Así, con el fin de añadir un nuevo tipo de
visualización solo se habrían de realizar las siguientes acciones:\\

\begin{enumerate}
		\item Crear un nuevo objeto que implemente los métodos necesarios
				de la clase \verb|Object|.
		\item Añadir un nuevo modo al \verb|enum Modes|.
		\item Añadir las opciones necesarias para dicho objeto en el programa
				principal e incluir la nueva clase \verb|#include "class.h"|.
		\item Actualizar las dependencias en el Makefile.
\end{enumerate}

En esta ventana principal se tiene por defecto un sistema de cámara en primera
persona, con la capacidad de moverse para visualizar mejor detalles del objeto
en cuestión. Este sistema puede ser sobreescribir en el objeto específico en
caso de necesitar otro comportamiento. \\

En la clase \verb|Object| existen tres métodos virtuales puros, que han de ser
implementados por las clases específicas de cada tipo de visualización:

\begin{itemize}
		\item \verb|draw()|, que ha de encargarse de dibujar el objeto en
				cuestión.
		\item \verb|processInputGLFWwindow * window)|, que ha de especificar que
				hacer con la entrada del usuario para este tipo de
				visualización.
		\item \verb|setUniforms()|, que ha de especificar las variables
				\verb|uniform| que utilicen los shaders de este tipo de
				visualización.
\end{itemize}

Estos métodos se llaman una vez por vuelta del bucle principal. En la sección
siguiente se introducen las matemáticas necesarias e importantes para el
desarrollo de la aplicación y los shaders concretos.

\section{Matemáticas necesarias}
\label{makereference5.4}

Con el fin de desarrollar el sistema de cámaras que utiliza la aplicación es
necesario conocer varios conceptos importantes sobre álgebra lineal y
transformaciones matriciales, así como los ángulos de Euler y su relación con
los cuaterniones. También se explorarán distintos métodos numéricos, relevantes
en el método de Line Integral Convolution, así como fórmulas matemáticas
específicas de cada tipo de visualización.

\subsection{Transformaciones matriciales}
\label{makereference5.4.1}
\subsection{Ángulos de Euler y Cuaterniones}
\label{makereference5.4.2}
\subsection{Métodos Numéricos para la resolución de Ecuaciones Diferenciales}
\label{makereference5.4.3}
\subsection{Otras Fórmulas}
\label{makereference5.4.4}

\section{Shaders en la aplicación}
\label{makereference5.5}

\subsection{Coloreado de terrenos}
\label{makereference5.5.1}
\subsection{Curvas de Bézier}
\label{makereference5.5.2}
\subsection{Superficies de Bézier}
\label{makereference5.5.3}
\subsection{Sólidos de revolución}
\label{makereference5.5.4}
\subsection{Nube de puntos}
\label{makereference5.5.5}
\subsection{Negativo de una imagen}
\label{makereference5.5.6}
\subsection{Detección de bordes}
\label{makereference5.5.7}
\subsection{Line Integral Convolution}
\label{makereference5.5.8}
