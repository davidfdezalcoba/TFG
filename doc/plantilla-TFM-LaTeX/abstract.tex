% +--------------------------------------------------------------------+
% | Copyright Page
% +--------------------------------------------------------------------+

\newpage

\thispagestyle{empty}

\begin{center}

{\bf \Huge Abstract}

  \end{center}
\vspace{1cm}

	Scientific visualization has traditionally been running on CPUs. However, we
	live in a time where GPUs are improving at a drastic rate, making a
	transition toward the parallelization capabilities that GPUs provide seem
	natural.  Within this context, graphic shaders, introduced in graphics cards
	and programmable, give very interesting features for us to use in scientific
	visualization. This text analyzes and describes the various properties of
	each of the shaders in the graphics pipeline, applying them to several
	well-known scientific visualization techniques. Some of the techniques
	covered involve image transformation, point cloud for visualizing 3D scalar
	data, height coloring terrains, bézier curves and surfaces or hedgehog plots
	for 3D vectorial data.

\vspace{1cm}

% +--------------------------------------------------------------------+
% | On the line below, replace Fecha
% |
% +--------------------------------------------------------------------+

\begin{center}

{\bf \Large Keywords}

   \end{center}

   \vspace{0.5cm}
   
	Scientific Visualization, Graphic Shaders, GPU, Bézier
   


