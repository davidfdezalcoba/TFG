% +--------------------------------------------------------------------+ |
% Copyright Page
% +--------------------------------------------------------------------+

\newpage

\thispagestyle{empty}

\begin{center}

{\bf \Huge Resumen en castellano}

  \end{center} \vspace{1cm}

Las GPUs se utilizan normalmente para juegos y aplicaciones gráficas similares,
prestando menos atención a otros aspectos en los que pueden ser útiles, como lo
es la visualización científica. En este trabajo se analiza la utilidad de los
shaders para problemas de este tipo, demostrando la capacidad de cada uno de sus
tipos (vertex, tesselation, geometry, fragment). Además, se introducen ejemplos
de conjuntos de datos y modelos reales para ilustrar su capacidad, concretamente
utilizando técnicas de visualización de imágenes, modelos y renderizado de
superficies. El resultado es una aplicación de visualización científica en la
que el usuario puede cargar sus propios modelos y aplicar sobre ellos algunas de
las técnicas propuestas. Como posible trabajo futuro, se propone la extensión de
la aplicación con otras de las técnicas conocidas de visualización científica.

\vspace{1cm}

% +--------------------------------------------------------------------+ | On
% the line below, replace Fecha |
% +--------------------------------------------------------------------+

\begin{center}

{\bf \Large Palabras clave}

   \end{center}

   \vspace{0.5cm}
   
   Lista de palabras clave
   


