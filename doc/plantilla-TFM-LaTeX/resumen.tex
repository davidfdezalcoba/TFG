% +--------------------------------------------------------------------+ |
% Copyright Page
% +--------------------------------------------------------------------+

\newpage

\thispagestyle{empty}

\begin{center}

{\bf \Huge Resumen}

  \end{center} \vspace{1cm}

    Tradicionalmente, la visualización científica se ha realizado en CPUs. Sin
    embargo, vivimos unos días en los que las GPUs están mejorando
    drásticamente, por lo que resulta natural una transición de la visualización
    científica hacia entornos que utilicen las capacidades de paralelización que
    nos proporcionan las GPUs. En este contexto, los shaders gráficos,
    introducidos en las tarjetas gráficas y programables, nos aportan
    capacidades muy interesantes para este ámbito. Así, en este texto se
    analizan y describen las capacidades de cada uno de los tipos de shaders del
    pipeline de renderizado aplicadas a técnicas de visualización tradicionales,
    como transformaciones de imágenes, nubes de puntos para datos escalares 3D,
    visualización de alturas de terrenos, curvas y superficies de bézier o
    hedgehog plots para datos vectoriales en 3D.

\vspace{1cm}

% +--------------------------------------------------------------------+ | On
% the line below, replace Fecha |
% +--------------------------------------------------------------------+

\begin{center}

{\bf \Large Palabras clave}

   \end{center}

   \vspace{0.5cm}
   
   Visualización científica, GPU, Shader, Bézier
   


